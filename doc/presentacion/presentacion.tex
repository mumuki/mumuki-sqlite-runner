\documentclass{beamer}

\usepackage[utf8]{inputenc}
\usepackage[spanish,activeacute,es-noshorthands]{babel}

\usetheme{Warsaw}

\usepackage{graphicx}
\graphicspath{ {../imgs/} }



\title[Mumuki SQL]
{%
    \normalsize{Presentación \\ Trabajo de Inserción Profesional}%
}

\author[Leandro Di Lorenzo]
{
    Leandro~Di~Lorenzo
}


\institute[TIP - UNQ]
{
    \textsc{
        \normalsize{Universidad Nacional de Quilmes}
    }
}

\date[apr 2017]{1 de Abril, 2017}


\makeatletter
    \newenvironment{withoutheadline}{
        \setbeamertemplate{headline}[default]
        \def\beamer@entrycode{\vspace*{-\headheight}}
    }{}
\makeatother


\begin{document}

\begin{withoutheadline}
    \begin{frame}
        \vspace{15pt}
        \begin{figure}[h]
                \includegraphics[scale=1]{logo-mumuki-sql}
        \end{figure}
        \vspace{-10pt}
        \titlepage
    \end{frame}
\end{withoutheadline}


\begin{frame}{Bla bla}
    \tableofcontents[pausesections]
\end{frame}


%\section{Qué es y para qué sirve la Autenticación}
%
%\begin{frame}{}
%    \frametitle{Qué es y para qué sirve la Autenticación}
%    \begin{block}{Autenticación}
%        Es un mecanismo que permite verificar la autenticidad de algo.
%    \end{block}
%    \pause
%    \vspace{2em}
%    \begin{exampleblock}{Identidades}
%        Particularmente nos va a interesar la autenticidad de las identidades de las personas.
%    \end{exampleblock}
%\end{frame}
%
%\begin{frame}{}
%    \frametitle{Ejemplos de Autenticación de Personas}
%    \begin{itemize}[<+->]
%        \item En compras con tarjeta de crédito se pide el documento para verificar que el titular sea el de la foto
%        \item El acceso a un datacenter se realiza mediante un sensor de huella dactilar
%        \item Un perito calígrafo compara y detecta validez de firmas en documentos
%        \item Dibujar un patrón sobre puntos permite desbloquear el celular
%        \item Una puerta cerrada con llave... \onslide<+-> solo pueden entrar los que tengan la llave
%        \item ¿Se les ocurren otros?
%    \end{itemize}
%\end{frame}
%
%\begin{frame}{}
%    \frametitle{Otras consideraciones sobre la Autenticación}
%    \begin{itemize}[<+->]
%        \item Puede otorgar roles y permisos según el autenticado
%        \item En general nos va a interesar la validación de \emph{usuario} y \emph{contraseña}
%        \item Es importante resguardar las credenciales de autenticación:
%        \begin{itemize}
%            \item Si los valores quedan expuestos el sistema queda expuesto
%            \item Se pierde la garantía de identidad
%        \end{itemize}
%    \end{itemize}
%\end{frame}
%
%
%\begin{frame}{}
%    \frametitle{Protocolo básico}
%    \begin{figure}[h]
%        \includegraphics[width=\textwidth]{imgs/auth-protocol}
%    \end{figure}
%\end{frame}
%
%\begin{frame}{}
%    \frametitle{\emph{Stateful}, \emph{Stateless}}
%    \begin{block}{Stateful}
%        Una vez que el usuario es autenticado esa información
%        se mantiene en el \emph{estado} de la aplicación.
%    \end{block}
%    \pause
%    \vspace{2em}
%    \begin{exampleblock}{Stateless}
%        Como no hay \emph{estado}, en cada petición es necesario validar
%        la autenticidad de la persona.
%    \end{exampleblock}
%\end{frame}
%
%\section{Sesión vs Token (Based Authentication)}
%\begin{frame}{}
%    \frametitle{Session Based Auth}
%    \begin{itemize}[<+->]
%        \item Una vez que el usuario se autentica el servidor genera una sesión
%        para el usuario y la almacena en memoria
%        \item El servidor responde al usuario con el id de sesión
%        \item El usuario debe guardar el id de la sesión en una cookie para mantenerla
%        durante el tiempo que use la aplicación
%        \item En cada petición el usuario debe mandar el id de la sesión
%    \end{itemize}
%\end{frame}
%
%\begin{frame}{}
%    \frametitle{Peticiones en Session Based}
%    \begin{figure}[h]
%        \includegraphics[width=\textwidth]{imgs/session-protocol}
%    \end{figure}
%\end{frame}
%
%\begin{frame}{}
%    \frametitle{Token Based Auth}
%    \begin{itemize}[<+->]
%        \item Una vez que el usuario se autentica el servidor genera un token
%        con los datos del usuario y determinadas configuraciones de seguridad
%        \item El servidor responde enviando el token generado
%        \item El servidor no almacena el token
%        \item El usuario puede guardar el token en cualquier lado
%        \item Las cookies son una forma más de almacenamiento
%        \item En cada petición el usuario debe mandar el token
%        \item Se puede mandar por Cookie, Header, GET, POST, etc...
%        \item Si existe más de un servidor que requiera usuarios autenticados,
%        todos pueden decodificar el token sin almacenarlo
%    \end{itemize}
%\end{frame}
%
%\begin{frame}{}
%    \frametitle{Peticiones en Token Based}
%    \begin{figure}[h]
%        \includegraphics[width=\textwidth]{imgs/token-protocol}
%    \end{figure}
%\end{frame}
%
%\begin{frame}
%  \frametitle{Lecturas Recomendadas}
%  \begin{thebibliography}{10}
%  \beamertemplatearticlebibitems
%  \bibitem{ins-outs}
%    Chris Sevilleja, Scotch.io
%    \newblock The Ins and Outs of Token Based Authentication
%    \newblock \url{https://scotch.io/tutorials/the-ins-and-outs-of-token-based-authentication}
%
%  \bibitem{cookie-token}
%    Alberto Pose, Auth0
%    \newblock Cookies vs Tokens. Getting auth right with Angular.JS
%    \newblock \url{https://auth0.com/blog/2014/01/07/angularjs-authentication-with-cookies-vs-token/}
%
%  \bibitem{http-bad}
%    Adrian Otto
%    \newblock Why HTTP Basic Auth is Bad
%    \newblock \url{http://adrianotto.com/2013/02/why-http-basic-auth-is-bad/}
%
%  \end{thebibliography}
%\end{frame}
%
%\section{OAuth 2.0: The Social Media Authorization}
%\begin{frame}{}
%    \begin{center}\begin{LARGE}
%        OAuth 2.0: The Social Media Authorization
%    \end{LARGE}\end{center}
%\end{frame}
%
%\section{JSON Web Tokens: The Open Token Solution}
%\begin{frame}{}
%    \begin{center}\begin{LARGE}
%        JSON Web Tokens: The Open Token Solution
%    \end{LARGE}\end{center}
%\end{frame}
%
%\section{Demo JSON Web Tokens}
%\begin{frame}{}
%    \begin{center}\begin{LARGE}
%        Demo JSON Web Tokens
%    \end{LARGE}\end{center}
%\end{frame}


\end{document}
