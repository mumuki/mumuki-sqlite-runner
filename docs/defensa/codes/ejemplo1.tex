Ejemplo de un ejercicio con solución vía \textit{Query}

\begin{columns}[t]
    \column{.5\textwidth}
    \begin{listing}[H]
        \caption{Extra (Doc)}
        \begin{minted}[fontsize=\scriptsize]{sql}
CREATE TABLE motores (
    id integer primary key,
    nombre varchar(200) NOT NULL
);
        \end{minted}
    \end{listing}

    \begin{listing}[H]
        \caption{Content (Alu)}
        \begin{minted}[fontsize=\scriptsize]{sql}
        SELECT id, nombre
        FROM motores;
        \end{minted}
    \end{listing}

    \column{.5\textwidth}
    \begin{listing}[H]
        \caption{Test (Doc)}
        \begin{minted}[fontsize=\tiny]{yaml}
solution_type: "query"
solution_query: "select * from motores";
examples:
  - data: |
      INSERT INTO motores values ('MySQL');
      INSERT INTO motores values ('PostgreSQL');
      INSERT INTO motores values ('Oracle');
      INSERT INTO motores values ('SQL Server');
      INSERT INTO motores values ('SQLite');
        \end{minted}
    \end{listing}

\end{columns}
